\chapter{Concluding Remarks}

In this thesis novel techniques to investigate the non-stationarity of electrophysiological connectivity of the human brain have been described. We have demonstrated that via the exploitation of the excellent temporal resolution of MEG, it has been possible to push the temporal scale on which functional connectivity can be assessed from minutes and hours, to seconds. Specifically, we took two differing approaches to assessing the dynamic connectome. In Chapters 5 and 6 we used canonical correlation to investigate connections within previously established networks, at high spatiotemporal reosultion. We found that the large canonical network was in fact a temporal aggregate of many smaller subnetworks, all of which would rapidly form and dissolve based on cognitive demand. Chapter 7 took a somewhat different approach, instead opting for an all-to-all connectivity, by simultaneously testing connections between many different node pairs. This allowed for better coverage of the brain volume, albeit at the expense of spatial resolution. Combining this with temporal ICA allowed us to discriminate networks by their temporal connectivity signature. Overall, the methods developed and published as part of this thesis offer a novel means to assess dynamic electrophysiological connectivity. These methods will be of significant utility in the future study of brain function.

\section{Future directions}

The most obvious future direction would be to take the work perfomed here into the clinical domain. The methods developed were designed with this in mind; to assess and highlight the potential differences in functional connections in health and disease. As a research group we have a vested interest in investigating functional connectivity perturbations in patients with schizophrenia. The Nottingham laboratory has collected a large dataset of healthy controls and psychosis patients in a project known as the multimodal imaging study in psychosis (MISP). The data collected for MISP contain a cohort of around 40 patients who suffer from multiple forms of psychosis, including schizophrenia and Bipolar disorder, as well as a set of matched healthy controls. There are many sensory and cognitive tasks in the MEG data, from relevance modulation to visuomotor to resting state paradigms, which could lend themselves to a dynamic functional connectivity analysis. For example, it is hypothesised that salience is one of the key differences between those with schizophrenia \citep{Kapur2003}, and the healthy brain so we could assess functional connections using canonical correlation between regions of the bilateral insular network to see if there are any significant differences in the resting state or task modulated connectivity. Early results from this dataset show there is abnormal visuomotor processing in the patients from assessing amplitude of oscillatory power \citep{Robson2015}, so there is significant potential to find differences in the communication between the visual and motor regions.

As always, the work here is far from complete. That is to say, whilst this work may have immediate clinical application, there are also many alternative directions. The methods in Chapters 5 and 6 have quite different approaches to that in Chapter 7; they assess functional connections on different spatial scales and they have their own strengths and weaknesses. Canonical correlation (Chapters 5 and 6) boasts high spatial resolution but is ultimately limited to being a pairwise test, meaning we have to make an \textit{a-priori} selection on where to assess whilst neglecting a large proportion of the brain. The all-to-all approach instead trades spatial specificity for coverage of the entire brain. If we could fuse the two approaches, such that we take the high spatial resolution of CCA and combine it with the coverage of the all-to-all connections, we could extensively map many functional connections we haven't covered in this thesis. For example if we perform canonical correlation analysis between all parcel pairs (i.e we use all voxels rather than a single representative timecourse) we may find a middle-ground between the two. However this raises questions about how to best correct for leakage and whether we have the computing power to effectively deal with the amount of data this may potentially produce. Also, the use of an anatomical atlas to dictate the locations of our ROIs for all-to-all connections in Chapter 7 meant that some parcels may possibly contain multiple functional hubs which have been neglected. By using CCA to reveal the many sub networks in resting state networks, we could build a MEG specific, functional atlas on which to base our ROI selection in the future.

One technical consideration which applies to all experimental work presented here is that of source reconstruction in MEG. Here we have used a beamformer \citep{Robinson1999,Brookes2008} to project data from sensor to source space, and used it successfully reveal spatiotemporal dynamics of functional connectivity. However, we have (considering the dynamic nature of the findings) applied a static beamformer; that is to say the covariance and source orientation estimations were fixed for an entire experiments worth of data; assumptions which appear to be at odds with each other. The choice of a single covariance matrix spanning the entire experiment is based on the fact that errors in covariance estimation are reduced the more temporal data they are based on which reduces localisation errors \citep{Brookes2008}. However this may not account for some temporally non-stationary artefacts and so dynamic covariance matrices may be required. Additionally we know that the space represented by a single point in MEG space is consists of multiple sources possibly in many orientations, which may show maximal variance at different time and so the assumption of stationary orientation is unfounded. Some methods already exist for dynamic source reconstruction based on time evolving covariance exist \citep{Dalal2008,Woolrich2013}, and so it could be suggested that for future work that these methods may perhaps be implemented for improved localisation of dynamic connections. 

Finally, one topic which has not been covered at all within this thesis is the study of effective connectivity. Effective connectivity can be defined as the \textit{influence one neural system exerts over another} \citep{Friston1994}. In short, effective connectivity theory states that functional connections may have a direction associated with them, and we seek to assess in which direction this neural information flows. Many methods to assess this exist, such as autoregressors, Granger causality \citep{Granger1969}, Dynamic Causal Modelling (DCM; \citealp{Friston2003}) or the Phase Slope Index (PSI; \citealp{Nolte2008}). The field of fMRI has shown interest in investigating effective connections, but in the same way dynamic functional connectivity, it is ultimately held back by fMRI's poor temporal resolution. MEG, with its temporally rich data is better suited to utilise these methods. In particular the all-to-all connectivity methods described in Chapter 7 would be a perfect candidate for Granger causality or PSI to be used in instead of amplitude envelope correlations. If we allow these directions to also time evolve, we would ultimately have 6 dimensions of functional connectivity to exploit. 

\section{Epilogue}
Overall, the field of dynamic connectivity remains in its infancy. However new methods, including those presented here, are currently emerging which begin to allow researchers a window on how the brain dynamically forms and dissolves networks based on current demand. The further development of these methods and their application to basic and clinical neuroscience, has the potential to generate a fundamentally new understanding of human brain function, and its breakdown in disease.  